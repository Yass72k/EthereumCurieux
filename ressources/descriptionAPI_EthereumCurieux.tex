\documentclass[a4paper,11pt]{article} 
\usepackage[francais]{babel}
\usepackage[T1]{fontenc} 
\usepackage[utf8]{inputenc} 
\usepackage{graphicx}
\usepackage{color}
\usepackage{hyperref}
\usepackage{bookmark}
\usepackage{array}
\usepackage{geometry}
\geometry{a4paper, margin=1in}

\begin{document}

\begin{figure}
\includegraphics[width=0.3\textwidth]{images/logolemansU.png}
\hspace{150pt} 
\includegraphics[width=0.3\textwidth]{images/logo_ic2.png} 
\end{figure}

\title{\textbf{\color{blue} Le Mans Universit\'e}\color{black}
\\ Licence Informatique \textit{3\`eme ann\'ee}
\\ Module IHM \& WEB
\\ Ethereum Curieux 
\\ \textbf{Description des API}} 
\author{LEPINE Francois\\BORRY Lenny\\GHRIB Yacine\\VEAU-BIGOT Damien}
\date{\today} 
\maketitle 
\newpage

\tableofcontents
\newpage

\section{Introduction}

\section{APIs Utilis\'ees}
Dans ce projet, plusieurs APIs ont été utilisées pour fournir des informations relatives aux crypto-monnaies et aux actualités. Voici une description des APIs principales :

\subsection{API pour les donn\'ees des crypto-monnaies}
Nous avons utilis\'e des services d'API permettant de r\'ecup\'erer les informations et les donn\'ees en temps r\'eel concernant les crypto-monnaies. Parmi les options disponibles, nous avons choisi \textbf{CoinGecko API} pour les raisons suivantes :

\begin{itemize}
    \item \textbf{Gratuit\'e et accessibilit\'e} : CoinGecko API offre un acc\`es plus ouvert aux donn\'ees essentielles, contrairement \`a CoinMarketCap API qui impose des restrictions plus strictes sur les appels gratuits.\
    \item \textbf{Richesse des donn\'ees} : Elle fournit des informations d\'etaill\'ees sur plus de 13 000 crypto-monnaies, incluant les prix en temps r\'eel, la capitalisation boursi\`ere, le volume d'\'echange et des graphiques historiques.\
    \item \textbf{Absence de cl\'e API obligatoire} : Contrairement \`a CoinMarketCap, qui exige une inscription et une cl\'e API, CoinGecko permet de r\'ecup\'erer des informations sans cl\'e API, simplifiant ainsi l’int\'egration.\
    \item \textbf{Fiabilit\'e et popularit\'e} : Utilis\'ee par de nombreuses plateformes de trading, CoinGecko est une source de donn\'ees stable et r\'eput\'ee.
\end{itemize}

Vous pouvez consulter la documentation officielle \`a l'adresse suivante : \href{https://www.coingecko.com/en/api}{CoinGecko API}.

\subsection{API pour les actualit\'es}
Pour obtenir les derni\`eres nouvelles en rapport avec les crypto-monnaies, nous avons choisi \textbf{NewsAPI}, en raison des avantages suivants :

\begin{itemize}
    \item \textbf{Diversit\'e des sources} : NewsAPI agr\'ege des articles de plusieurs m\'edias reconnus, garantissant une couverture compl\`ete de l’actualit\'e crypto.\
    \item \textbf{Flexibilit\'e des requ\^etes} : Elle permet de filtrer les articles par mot-cl\'e, source, date et langue, offrant un contr\^ole pr\'ecis sur les informations r\'ecup\'er\'ees.\
    \item \textbf{Facilit\'e d’int\'egration} : L’API est bien document\'ee et simple \`a utiliser, permettant une int\'egration rapide.\
    \item \textbf{Co\^ut abordable} : Un plan gratuit permet d’acc\'eder aux titres et r\'esum\'es des articles, suffisants pour nos besoins.
\end{itemize}

La documentation de NewsAPI est disponible ici : \href{https://newsapi.org/}{NewsAPI}.

\subsection{Conclusion}
Le choix de \textbf{CoinGecko API} et \textbf{NewsAPI} repose sur leur \textbf{accessibilit\'e}, \textbf{richesse en informations}, \textbf{fiabilit\'e} et \textbf{simplicit\'e d’int\'egration}. Ces APIs permettent d’obtenir des donn\'ees en temps r\'eel sur les crypto-monnaies et leurs actualit\'es, tout en optimisant les co\^uts et la facilit\'e d’utilisation.


Ces APIs ont été choisies pour leur fiabilité et leur richesse en termes de fonctionnalités.

\end{document}
