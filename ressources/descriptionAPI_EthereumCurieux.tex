\documentclass[a4paper,11pt]{article} 
\usepackage[francais]{babel}
\usepackage[T1]{fontenc} 
\usepackage[utf8]{inputenc} 
\usepackage{graphicx}
\usepackage{color}
\usepackage{hyperref}
\usepackage{bookmark}
\usepackage{array}
\usepackage{geometry}
\geometry{a4paper, margin=1in}

\begin{document}

\begin{figure}
\includegraphics[width=0.3\textwidth]{images/logolemansU.png}
\hspace{150pt} 
\includegraphics[width=0.3\textwidth]{images/logo_ic2.png} 
\end{figure}

\title{\textbf{\color{blue} Le Mans Universit\'e}\color{black}
\\ Licence Informatique \textit{3\`eme ann\'ee}
\\ Module IHM \& WEB
\\ Ethereum Curieux 
\\ \textbf{Description des API}} 
\author{LEPINE Francois\\BORRY Lenny\\GHRIB Yacine\\VEAU-BIGOT Damien}
\date{\today} 
\maketitle 
\newpage

\tableofcontents
\newpage

\section{Introduction}

\section{APIs Utilis\'ees}
Dans ce projet, plusieurs APIs ont été utilisées pour fournir des informations relatives aux crypto-monnaies et aux actualités. Voici une description des APIs principales :

\subsection{API pour les données des crypto-monnaies}
Nous avons utilisé des services d'API permettant de récupérer les informations et les données en temps réel concernant les crypto-monnaies. Voici deux exemples d'API populaires :

\begin{itemize}
    \item \textbf{CoinGecko API} : Cette API fournit des informations complètes sur les prix, les volumes, et les capitalisations des différentes crypto-monnaies. Elle permet également de récupérer les tendances actuelles et des graphiques historiques. Vous pouvez consulter la documentation sur leur site officiel : \href{https://www.coingecko.com/en/api}{CoinGecko API}.

    \item \textbf{CoinMarketCap API} : Un service d'API très populaire qui offre des données similaires à CoinGecko, avec des fonctionnalités avancées comme des indicateurs de marché et des analyses de performance. Plus de détails sont disponibles à l'adresse suivante : \href{https://coinmarketcap.com/api/}{CoinMarketCap API}.
\end{itemize}

\subsection{API pour les actualités}
Afin de fournir les dernières nouvelles en rapport avec les crypto-monnaies, nous avons utilisé des APIs pour les actualités :

\begin{itemize}
    \item \textbf{NewsAPI} : Une API générale pour récupérer des actualités provenant de différentes sources. Elle permet de filtrer les nouvelles selon des mots-clés ou des sources précises. Consultez leur documentation ici : \href{https://newsapi.org/}{NewsAPI}.

    \item \textbf{CryptoNews API} : Spécialisée dans les actualités des crypto-monnaies, cette API fournit des articles et des analyses provenant de plateformes reconnues dans le domaine. Vous pouvez consulter leur documentation sur \href{https://cryptonews-api.com/}{CryptoNews API}.
\end{itemize}

Ces APIs ont été choisies pour leur fiabilité et leur richesse en termes de fonctionnalités.

\end{document}
